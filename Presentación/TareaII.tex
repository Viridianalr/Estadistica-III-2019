\documentclass[11pt,spanish]{article}
\usepackage[margin=1in]{geometry} 
\usepackage[spanish]{babel}
\decimalpoint
\usepackage[utf8]{inputenc}
\usepackage{amsmath,amsthm,amssymb,hyperref}
\title{Estadística III: Inferencia no paramétrica para funciones de distribución}
\author{Alejandro López Hernández}
\begin{document}
\maketitle
\setlength\parindent{0pt}
\textbf{E1} Revisar el capitulo 26 de Anirban DasGupta, Asymptotic Theory of Statistics and Probability\cite{DaGuspa}.\\
\textbf{E2} Revisar el capitulo 2 de Wasserman, All of Nonparemetric Statistics \cite{Wasserman}.\\
\textbf{E3} Revisar el capitulo 3 y 4  de Hollander, Nonparemetric Statistical Methods \cite{Hollander}.\\\\
\textbf{Distribución Empírica}\\
\textbf{E4} Supongamos que tenemos los siguientes datos $X$ =12.2, 34, 41, 3.23, 24, -11, 0.12, 0.23, 23, 3, 5.89, 3, 23, 34 calcular lo siguiente: \\
\hspace*{6mm} a) $\hat{F}(3)$, $\hat{F}(32)$, $\hat{F}(-3)$, $\hat{F}(3.2)$, $\hat{F}(89)$
\hspace*{6mm} b) $\mathbb{P}(1<X<17)$, $\mathbb{P}(2<X^2<4)$, $\mathbb{P}(4<e^X<67)$\\
\hspace*{6mm} c) Encontrar las bandas de confianza de $\hat{F}$ 
\hspace*{6mm} d) Hacer un dibujo de $\hat{F}$ \\
\textbf{E5} Supongamos que tenemos los siguientes datos $Z = $(0,0), (0,1.2), (1,1), (0.23,0.45), (-1,1), (0.4,-0.12), (1.2,-0.2) calcular lo siguiente: \\
\hspace*{6mm} a) $\hat{F}(0,0)$, $\hat{F}(2,0)$, $\hat{F}(0,2)$, $\hat{F}(-1,0)$, $\hat{F}(-2,10)$\\
\hspace*{6mm} b) Sea $\mathbb{S}^2 = \{v\in\mathbb{R}^2:||v||<1\}$ calcular $\mathbb{P}(Z\in \mathbb{S}^2)$\footnote{$||\cdot||$ es la norma euclidiana}\\
\hspace*{6mm} c) $\mathbb{P}(Z\in [1,2]\times (0.23,1.1])$\\\\
\textbf{Inferencia No paramética}\\
\textbf{E6} Supongamos que tenemos los siguientes datos $X=$ 2, 3, 1, 4, 1.23, 5, 4, 6.2, 5, 7, 5, 4, 6.1, 3, 2.3, 1.6\\ Usar el estimador \emph{Plug-in} para calcular lo siguiente: \\
\hspace*{6mm} a) $T_1 = \int X dF$ \\
\hspace*{6mm} b) $T_2 = \int \sqrt{X} dF$ \\
\hspace*{6mm} c) $T_3 = \int \sin(X) dF$ \\
\hspace*{6mm} d) $T_4 =\int (X-T_1)^2 dF$ \\
\hspace*{6mm} e) $T_5 =\int (X-T_4)^4 dF$ \\
\hspace*{6mm} f) $T_6 =\int \frac{1}{1+X} dF$ \\
\hspace*{6mm} g) $T_7 =\int s^X dF$ \\
\hspace*{6mm} h) $T_8 =\inf\{x:F(x)\leq 0.75 \}$\\
\hspace*{6mm} i) $T_9 =\inf\{x:F(x)\leq 1 \}$\\
\textbf{E7} Usando el método delta no paramétrico encontar un intevalo de confianza para los siguientes funcionales de F \\
\hspace*{6mm} a) $T_1 =\int s^X dF$ \\
\hspace*{6mm} b) $T_2 = \int \sqrt{X} dF$ \\
\hspace*{6mm} c) $T_3 = \int \sin(X) dF$ \\\\
\textbf{Bondad de ajuste}\\
\textbf{E8} Sea la hipótesis $H_0: F = F_0(x)$ y los datos $X$=0.12 , 0.31 , 0.08, 0.067, 0.12, 0.344, 2.93, 0.31, 0.87, 0.10. Realizar una prueba de bondad de ajuste usando $D_n$, $A_n$, $C_n$ para los siguientes valores de $F_0$ ¿Cual es la más parecida a los datos?\\
\hspace*{6mm} a) $F_0=Normal(0,1)$ \\
\hspace*{6mm} b) $F_0=Exp(1)$\\
\hspace*{6mm} c) $F_0=Beta(1,1)$ \\
\hspace*{6mm} d) $F_0 =Unif(0,1)$ \\
\textbf{E9} Sea $X =$ 23, 22, 12, 31, 26, 19, 22 y $Y=$ 32, 34, 22, 18, 31, 25, 28 calcular $D_{n,m}$ de este par de distribuciones. \\\\
\textbf{Comparación de dos distribuciones}\\
\textbf{E10} Sea $R$ el estadístico de Mann-Whitney-Wilcoxon- si tenemos $n=3$ $m=2$ calcular su distribución $\mathbb{P}(R=k)$ para $k=3,4,5,6,7,8,9$\\
\textbf{E11} Sea $T$ el estadístico de rangos signados de Wilcoxon sea $n=3$ calcular la distribucion $\mathbb{P}(T=k)$ para $k=0,1,3,4,5,6$\\
\textbf{E12} Realizar una prueba de hipótesis para $H_0:\Delta =0$ y $H_1:\Delta \neq 0$ con el estadístico R para los siguientes datos: 
\begin{center}
\begin{tabular}{l|l}
$X$ & $Y$ \\ \hline \hline
12  & 3   \\ \hline
35  & 51  \\ \hline
31  & 23  \\ \hline
23  & 34  \\ \hline
9   & 26  \\ \hline
22  & 31  \\ \hline
41  & 6   \\ \hline
17  & 13  \\ \hline
21  & 39  \\ \hline
29  & 52  \\ \hline
14  &     \\ \hline
9   &     \\ \hline
59  &     \\ \hline
\end{tabular}
\end{center}
\textbf{E13} Realizar una prueba de hipótesis para $H_0:\theta =0$ y $H_1:\theta \neq 0$ con T para los siguientes datos: 
\begin{center}
\begin{tabular}{l||llllllllllll}
$X$ & 2.8 & 0.5 & 2.34 & 1.8 & 4.2 & 3.3 & 4.1 & 5.2 & 2.4 &1.2 & -0.3 &1.8\\ \hline 
$Y$ & 3 .4& 1.1 & 2.03 & 1.2 & 3.2 & 2.9 & 4.1 & 4.8 & 2.5 &1.25 & 0.1 &2.1\\ 
\end{tabular}
\end{center}
\begin{thebibliography}{9}
\bibitem{DaGuspa} 
Anirban DasGupta, \textit{Asymptotic Theory of Statistics and Probability}. 
First Edition, Springer 2008.
\bibitem{Wasserman} 
Larry Wasserman, \textit{All of Nonparemetric Statistics}. 
First Edition, Springer 2006.
\bibitem{Hollander} 
Myles Hollander, Douglas A. Wolfe y Eric Chicken, \textit{Nonparemetric Statistical Methods}. 
Third Edition, Wiley 2014.
\end{thebibliography}
\end{document}